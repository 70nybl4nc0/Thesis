\begin{opinion}
Replicar la inteligencia humana en una computadora es quizás el sueño último del campo de la inteligencia artificial, que tiene implicaciones tanto filosóficas como prácticas trascendentales. Por un lado, nos llevaría a una comprensión mucho más refinada de qué es eso que llamamos inteligencia. Por otro lado, abriría las puertas a aplicaciones nunca antes vistas, sustituyendo a los humanos en las tareas más complejas. Por tal motivo, la comunidad científica del campo de la IA, desde sus inicios, ha dedicado una inmensa cantidad de atención a definir mecanismos para evaluar el grado de inteligencia que posee un sistema computacional. Desde el famoso Test de Turing, hasta las variantes modernas de benchmarks de aprendizaje por refuerzo, son muchos los intentos por encontrar problemas suficientemente complejos que permitan decidir, sin lugar a dudas, que un sistema computacional efectivo en esos problemas es ``inteligente''.

El área de investigación en que incursiona el estudiante está relacionada con el diseño de un entorno de evaluación para agentes de aprendizaje por refuerzo que intenta capturar algunos de los aspectos más interesantes de la inteligencia general, tratando de introducir la menor cantidad de sesgos antropomórficos posible.

El  estudiante Tony Raúl Blanco Fernández en esta investigación se adentra en un tema del estado del arte de gran actualidad y para eso tuvo que utilizar conocimientos de varias asignaturas de la carrera y otros que no son parte del currículum estándar. Su propuesta implicó estudiar el estado del arte relacionado con la inteligencia en su definición más amplia, así como las arquitecturas computacionales, algoritmos de aprendizaje, y criterios de evaluación existentes en el dominio. Implementó así un entorno de evaluación interactivo para agentes de aprendizaje por refuerzo (computacionales y humanos) que es extensible a muchísimas tareas cognitivas presentadas en la literatura.

Sus resultados, aunque aún iniciales, son prometedores, pues las tareas implementadas muestran que los algoritmos de aprendizaje por refuerzo más comunes usados hoy día son incapaces de aprender efectivamente en estos entornos en el tiempo y con las restricciones diseñadas, mientras que los seres humanos son notablemente mejores. Todo esto demuestra que la inteligencia artificial general está lejos aún, a la vez que brinda una herramienta más para avanzar en ese sentido.

Para poder afrontar el trabajo, el estudiante tuvo que revisar literatura científica relacionada con la temática así como soluciones existentes y bibliotecas de software que pueden ser apropiadas para su utilización. Todo ello con sentido crítico, determinando las mejores aproximaciones y también las dificultades que presentan.

Todo el trabajo fue realizado por el estudiante con una elevada constancia, capacidad de trabajo y habilidades, tanto de gestión, como de desarrollo y de investigación.

Por estas razones pedimos que le sea otorgada al estudiante Tony Raúl Blanco Fernández la máxima calificación y, de esta manera, pueda obtener el título de Licenciado en Ciencia de la Computación.

\vspace{1cm}

\begin{center}
    Dr. Alejandro Piad Morffis\\
    Facultad de Matemática y Computación\\
    Universidad de La Habana\\
    Tutor\\
\end{center}


\end{opinion}

