\begin{resumen}
	Con el objetivo medir la . En el contexto de la interpretación y traducción de lenguas de señas, existen avances internacionalmente, pero ninguno relacionado con la Lengua de Señas Cubana. Debido a ello, es necesa- rio la realización de estudios y la implementación de plataformas que contribuyan a una mejor inserción en la sociedad de las personas que dependan de la Lengua de Señas Cubana para comunicarse. En este trabajo se establecen las pautas de investi- gación y los recursos primarios para el desarrollo a futuro de una plataforma para la interpretación automática de la Lengua de Señas Cubana.
\end{resumen}

\begin{abstract}
	Resumen en inglés
\end{abstract}