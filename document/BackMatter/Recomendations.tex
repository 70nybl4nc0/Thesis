\begin{recomendations}\label{recomendations}
    Este trabajo marca pautas en el desarrollo de un entorno de evaluación apto para algoritmos de aprendizaje por reforzamiento. En este sentido se avizora debilidades y futuras mejoras:
    
    \begin{itemize}
    \item La validez de UTHOPIA debe investigarse mediante estudios estadísticos de gran tamaño de muestra en humanos.
    \item Medir de forma cuantitativa la capacidad de generalización amplia.
    \item Crear y aumentar el conjunto de juegos de UTHOPIA y crear un conjunto privado de las mismas para hacerle frente a las estrategias de atajo que podrían resolver las tareas sin presentar inteligencia.
    \item Fundamentar los aspectos teóricos que componen los prejuicios y otros factores que influyen en el razonamiento humano, tales como la memoria asociativa y los conocimientos adquiridos.
    \item Automatizar el proceso de evaluación y permitir a investigadores probar sus algoritmos.
    \end{itemize}
    
\end{recomendations}
    
    