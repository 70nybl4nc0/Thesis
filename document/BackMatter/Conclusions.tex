\begin{conclusions}
    
    El objetivo fundamental de este trabajo fue establecer las pautas de investigación para el desarrollo a futuro de una plataforma para evaluar tareas de carácter cognitivo en algoritmos de aprendizaje por reforzamiento.
    
    Inicialmente se realizó un estudio del estado del arte de los métodos de evaluación de la generalización y la inteligencia en el aprendizaje de máquinas, analizando sus ventajas y sus problemas a partir de las señalaciones hechas por Chollet en (\cite{chollet2019measure}). Entre los aspectos a destacar en el estado del arte se encuentra la evidencia del auge de los entornos de evaluación de las capacidades de generalización en la adquisición de habilidades y en la búsqueda de evaluar sistemas con capacidades similares a las humanas en cuanto a razonamiento y eficiencia.
    
    Tomando como base los conocimientos del estado del arte se realizó una propuesta para la evaluación de inteligencia: UTHOPIA, para permitir las comparaciones en tareas cognitivas entre humanos y agentes inteligentes. Además se implementó un prototipo de dicha propuesta para mostrar su viabilidad utilizando Unity ML Toolkit como base.
    
    Se ejecutó un grupo de experimentos evidenciando en la propuesta realizada que los agentes humanos son en efecto capaces de dominar las variadas tareas eficientemente sin la necesidad de muchos datos de prácticas. A diferencia de los algoritmos PPO y SAC, los cuales exponen comportamientos similares a los aleatorios al ser entrenados con pocos datos. 
    
    \end{conclusions}
    
    