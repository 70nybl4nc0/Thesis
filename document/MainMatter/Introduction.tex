\chapter*{Introducción}\label{chapter:introduction}\addcontentsline{toc}{chapter}{Introducción}
La inteligencia tiene muchos matices. Desde simples sistemas capaces de adaptarse al ambiente, como plantas enredaderas al subir una pared, a los humanos capaces de llegar al espacio utilizando la ciencia. En las máquinas también se aprecia la inteligencia. Desde el punto donde son capaces de adaptarse ante situaciones nuevas, como lo hacen los algoritmos de reconocimiento de rostros o una computadora capaz de ganarle al campeón mundial de ajedrez. 

La idea de sistemas aprendiendo cómo resolver problemas mediante un proceso de aprendizaje similar al de los niños humanos no es nueva, más bien se remonta a los mismos orígenes de la computación, incluso Alan Turing creía que ese sería el futuro y en 1950 propuso varias pruebas para esos sistemas. Entre ellas la más conocida es \textit{El Juego de la imitación}, donde las computadoras se consideran inteligentes si lograban entablar una conversación indistinguible a como lo haría una persona. 

Lograr que las máquinas sean tan, o más, inteligentes que el hombre ha sido una de las temáticas de ciencia ficción más utilizada y vista por muchos como limitada al futuro lejano, pero en este los avances de este siglo indican lo contrario. Algoritmos como GPT-3 son capaces de escribir profundamente como un poeta o hacer historias cual escritor experimentado; otros como Dall-E 2 y Stable Difussion son casi una amenaza para el trabajo de pintores y artistas digitales. Incluso Github Copilot ayuda a escribir códigos como si fuera un programador asistente. Todo esto y mucho más son indicaciones de que estamos más cerca de crear una inteligencia similar a la humana.

La inteligencia es asociada, en sentido popular, a la capacidad que tiene un procesador (mecánico o biológico) de realizar complejos cálculos o cómputos para resolver problemas de forma eficiente, precisa y flexible. En este documento, al contrario, se aborda un concepto de inteligencia fundamentalmente diferente, donde se valora mucho más la eficiencia en la adquisición de nuevas habilidades que el nivel de maestría individual alcanzado en ellas. Es decir, la capacidad de un sistema para aprender a resolver nuevas tareas nunca antes vistas por él.

En el campo de la inteligencia artificial (IA) la capacidad de aprendizaje es el objeto de estudio principal. Algunos autores incluso definen la IA como: La ciencia y la ingeniería de hacer que las máquinas realicen tareas que nunca han visto y para las que no han sido preparadas de antemano [\cite{mccarthy1987generality}]. Esta capacidad se denomina generalización y consiste en utilizar datos para resolver problemas con datos diferentes de los que originalmente aprendió. Normalmente divididos en datos de entrenamiento y datos de prueba o comprobación. La evaluación de tareas específicas centrada en la habilidad ha sido adecuada para los sistemas cerrados que buscan la solidez en entornos que sólo presentan incógnitas conocidas, pero el desarrollo de sistemas capaces de manejar incógnitas desconocidas requiere la evaluación de sus habilidades en un sentido general.

La historia de la IA ha consistido en ascender lentamente por el espectro de la capacidad de generalización, comenzando con sistemas que en gran medida no mostraban generalización (IA simbólica) y evolucionando hacia sistemas robustos (aprendizaje automático) capaces de generalización local. Ahora estamos entrando en una nueva etapa, en la que buscamos crear sistemas flexibles capaces de una amplia generalización (por ejemplo, vehículos autoconducidos, asistentes de IA o robots con desarrollo cognitivo). 

Recientemente se ha incrementado el interés por sistemas inteligentes con mejores capacidades de generalizar. Grandes empresas han hecho de dichos sistemas el epicentro de su modelo de negocio. Por ejemplo: compañías automovilísticas han apostado al mercado de pilotos automáticos capaces de conducir de forma segura, otras compañías como Boston Dynamics ha creado robots capaces de ayudar a realizar mantenimientos y detectar fallos a ingenieros o participar en operaciones de rescate ante desastres naturales, también vale mencionar asistentes de voz como Siri o Alexa entre muchos otros. Todos ellos realizan tareas complejas formadas por sub-tareas diferentes y son capaces de manejar un gran número de situaciones desconocidas y aprender de estas aunque están aún lejos de expresar una inteligencia como la humana. Los grados de la capacidad de generalización serán formalizados en el próximo capítulo.

\section*{Motivación}
A medida que surgen sistemas más flexibles se hace necesario establecer métodos para evaluar su capacidad de generalización. Lamentablemente poco se ha hecho en lo que concierne al desarrollo de estas pruebas debido a la falta de consenso en el tema y la dificultad que representa realizar mediciones cuantificables de la inteligencia. Es importante destacar que la inteligencia es una construcción abstracta, basada en la teorías y fenómenos estadísticos, en lugar de una propiedad objetiva y directamente medible de forma individual como la puntuación obtenida en un test específico, pero todos concuerdan con que existe un factor común entre todas sus manifestaciones. Las pruebas psicométricas, por ejemplo, puntúan la inteligencia humana en diferentes escalas las cuales aunque no tienen absoluta coherencia entre todas, sí expresan un patrón repetible en los individuos evaluados. 

Las mediciones de inteligencia de las máquinas, se hacían con una perspectiva antropocéntrica. Donde el grado de expresión de inteligencia se mide principalmente por la similitud de estos sistemas con los humanos al resolver las mismas tareas. En ese sentido también hubo una tendencia a realizar evaluaciones de carácter psicométrico. La mayoría de las pruebas propuestas requerían de jurados humanos y eran costosamente repetibles o carecían de fidelidad en su veredicto debido a que eran criterios personales.

Hoy se tienen muchas pruebas de inteligencia diferentes para los sistemas más complejos. En especial los algoritmos de Aprendizaje por Reforzamiento los cuales aprenden con un método y condiciones similares a los seres vivos ya que están sometidos a entornos desconocidos y es su interacción continua la que va generando en ellos una base de conocimientos útiles para entenderlo. La evaluación de este tipo de algoritmos, sobre todo los capaces de generalizar al grado humano, son de nuestro interés y para ellos no existen suficientes pruebas de inteligencia similar a la humana, como se agumenta más adelante, que tengan un enfoque correcto o sustentación teórica.

\section*{Objetivos del trabajo}

\subsection*{Objetivo general}
Definir la evaluación de sistemas con aprendizaje por reforzamiento en tareas de propósito general, utilizando como bases lo expuesto por Chollet, e implementar una instancia de dicha propuesta en Unity.

\subsection*{Objetivos específicos}
\begin{itemize}
    \item Estudiar las bases de los algoritmos de aprendizaje por reforzamiento.
    \item Plantear las principales definiciones sobre las nociones de la inteligencia.
    \item Recapitular la historia de las propuestas de evaluación para sistemas con inteligencia similar a la humana.
    \item Profundizar en las definiciones que expone Chollet y sus recomendaciones a la hora de crear un entorno de evaluación para sistemas inteligentes.
    \item Plantear otras nociones a tener en cuenta para evaluar específicamente algoritmos de aprendizaje por reforzamiento.
    \item Proponer e implementar UTHOPIA, un entorno de evaluación de algoritmos con aprendizaje por reforzamiento para tareas de carácter cognitivo de propósito general.
    \item Realizar experimentos sobre UTHOPIA comparando el rendimiento de humanos contra los principales algoritmos proveídos en Unity ML en las tareas propuestas.
\end{itemize}

\section*{Estructura del artículo}
Este trabajo consta de 6 partes. La Introducción nos acerca al problema de la naturaleza de la inteligencia y la forma de percibir. Luego en el estado del arte agrupamos los principales conceptos necesarios para entender los algoritmos de interés así como las nociones principales de generalización e inteligencia. A continuación se recorrerá el camino del desarrollo de evaluaciones en sistemas de gran inteligencia, incluidos los de aprendizaje por reforzamiento y la propuesta de Chollet, el ARC. En el siguiente capítulo, se hará una propuesta para un entorno de evaluación llamado UTHOPIA y se realizarán experimentos entre varios algoritmos y humanos. Concluirá el artículo listando una serie de recomendaciones para futuros trabajos con UTHOPIA y se condensarán las conclusiones tomadas a lo largo de toda labor. 


