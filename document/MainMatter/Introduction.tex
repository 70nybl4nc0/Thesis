\chapter*{Introducción}\label{chapter:introduction}
\addcontentsline{toc}{chapter}{Introducción}


cita: "La IA es la ciencia y la ingeniería de hacer que las máquinas realicen tareas que nunca han visto y para las que no han sido preparadas de antemano" John McCarthy. Generality in artificial intelligence. Communications of the ACM, 30(12):1030–1035, 1987.

La idea de que las maquinas podrian adquirir nuevas habilidades para resolver problemas mediante un proceso de aprendizaje similar al de los niños humanos se remonta incluso al los origenes del campo.  

La evaluación de tareas específicas centrada en las habilidades ha sido adecuada para los sistemas cerrados que buscan la solidez en entornos que sólo presentan incógnitas conocidas, pero el desarrollo de sistemas capaces de manejar incógnitas desconocidas requiere la evaluación de sus habilidades en un sentido general y en torno esta capacidad se desarrollará este documento.

La capacidad generalización se basa fundamentalmente en las nociones relacionadas a novedad e incertidumbre. Un sistema sólo puede generalizar ante información nueva que no pueda ser conocida de antemano ni por el sistema o por su creador. Dica capacidad puede presentarse en diferene a medida que aumenta el dominio de las tareas y problemas que se pueden manejar. Más adelante se profundizará más en su categorización.

La historia de la IA ha consistido en ascender lentamente por el espectro de esta capacidad, comenzando con sistemas que en gran medida no mostraban generalización (IA simbólica) y evolucionando hacia sistemas robustos (aprendizaje automático) capaces de generalización local. Ahora estamos entrando en una nueva etapa, en la que buscamos crear sistemas flexibles capaces de una amplia generalización (por ejemplo, vehículos autoconducidos, asistentes de IA o robots de desarrollo cognitivo).


describe la inteligencia como la eficiencia en la adquisición de habilidades 


Tipos de Generalizacion

- Generalización centrada en el sistema: es la capacidad de un sistema de aprendizaje para manejar situaciones que no ha encontrado antes. La noción formal de error de generalización en la teoría del aprendizaje estadístico pertenecería aquí.

- Generalización consciente del desarrollador: es la capacidad de un sistema, ya sea de aprendizaje o estático, para manejar situaciones que ni el sistema ni el desarrollador del sistema han encontrado antes.

Grados de Generalizacion

- Ausencia de generalización: Los sistemas de IA en los que no hay incertidumbre no muestran generalización. Por ejemplo, no se puede decir que un programa que juega a "4 en Línea" mediante una iteración exhaustiva "generalice" a todas las configuraciones del tablero.

- Generalización local o "robustez": Es la capacidad de un sistema para manejar nuevos puntos de una distribución conocida para una sola tarea o un conjunto bien delimitado de tareas conocidas, dado un muestreo suficientemente denso de ejemplos de la distribución (por ejemplo, la tolerancia a las perturbaciones previstas dentro de un contexto fijo). Por ejemplo, se puede decir que un clasificador de imágenes que puede distinguir imágenes RGB de 150x150 no vistas anteriormente que contienen gatos de las que contienen perros, después de haber sido entrenado con muchas de esas imágenes etiquetadas, realiza una generalización local. 

- Generalización amplia o "flexibilidad": Es la capacidad de un sistema para manejar una amplia categoría de tareas y entornos sin más intervención humana. Esto incluye la capacidad de manejar situaciones que no podrían haber sido previstas por los creadores del sistema. Podría considerarse que refleja la capacidad del ser humano en un único y amplio ámbito de actividad (por ejemplo, las tareas domésticas o la conducción en el mundo real).

- Generalización extrema: Describe los sistemas abiertos con la capacidad de abordar tareas completamente nuevas que sólo comparten puntos comunes abstractos con situaciones previamente encontradas, aplicables a cualquier tarea y dominio dentro de un amplio alcance. Esto podría caracterizarse como "adaptación a incógnitas desconocidas en una gama desconocida de tareas y dominios". Las formas biológicas de inteligencia (los humanos y posiblemente otras especies in- telligentes) son el único ejemplo de un sistema de este tipo en este momento.

A esta lista podríamos, en teoría, añadir una entrada más: La "universalidad", que extendería la "generalidad" más allá del ámbito de las tareas relevantes para los humanos, a cualquier tarea que pueda ser abordada de forma práctica dentro de nuestro universo (nótese que esto es diferente de "cualquier tarea en absoluto", tal y como se entiende en los supuestos del teorema No Free Lunch [98, 97]).



Es importante destacar que el espectro de la generalización descrito anteriormente parece reflejar la organización de las capacidades cognitivas de los seres humanos tal y como se establece en las teorías de la estructura de la inteligencia en la psicología cognitiva.



- Inteligencia. Inteligencia General "El santo grial de la IA"

- Rendimiento y Generalización. Key concepts.

- Primeras propuestas de evaluación, enfoque antropocéntrico de Turing. La prueba del café.

- Chollet, ARC 2019, formalizacion de las evaluaciones.

- Aprendizaje por reforzamiento y agentes.

- Vació existente de entornos de evaluación de IG. (motivacion?)