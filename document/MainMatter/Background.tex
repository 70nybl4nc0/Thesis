\chapter{Estado del Arte}\label{chapter:state-of-the-art}
La capacidad generalización se basa fundamentalmente en las nociones relacionadas a novedad e incertidumbre. Un sistema sólo puede generalizar ante información nueva que no pueda ser conocida de antemano ni por el sistema o por su creador. Dica capacidad puede presentarse en diferene a medida que aumenta el dominio de las tareas y problemas que se pueden manejar. Más adelante se profundizará más en su categorización.

La comunidad contemporánea de la IA todavía se inclina por evaluar la inteligencia comparando la destreza que muestran las IA y los humanos en tareas específicas, como los juegos de mesa y los videojuegos. Para entender la diferencia entre la eficiencia y la capacidad de generalizar se definirá varios conceptos a continuacion:

Tipos de Generalizacion

- Generalización centrada en el sistema: es la capacidad de un sistema de aprendizaje para manejar situaciones que no ha encontrado antes. La noción formal de error de generalización en la teoría del aprendizaje estadístico pertenecería aquí.

- Generalización consciente del desarrollador: es la capacidad de un sistema, ya sea de aprendizaje o estático, para manejar situaciones que ni el sistema ni el desarrollador del sistema han encontrado antes.

Grados de Generalizacion

- Ausencia de generalización: Los sistemas de IA en los que no hay incertidumbre no muestran generalización. Por ejemplo, no se puede decir que un programa que juega a "4 en Línea" mediante una iteración exhaustiva "generalice" a todas las configuraciones del tablero.

- Generalización local o "robustez": Es la capacidad de un sistema para manejar nuevos puntos de una distribución conocida para una sola tarea o un conjunto bien delimitado de tareas conocidas, dado un muestreo suficientemente denso de ejemplos de la distribución (por ejemplo, la tolerancia a las perturbaciones previstas dentro de un contexto fijo). Por ejemplo, se puede decir que un clasificador de imágenes que puede distinguir imágenes RGB de 150x150 no vistas anteriormente que contienen gatos de las que contienen perros, después de haber sido entrenado con muchas de esas imágenes etiquetadas, realiza una generalización local. 

- Generalización amplia o "flexibilidad": Es la capacidad de un sistema para manejar una amplia categoría de tareas y entornos sin más intervención humana. Esto incluye la capacidad de manejar situaciones que no podrían haber sido previstas por los creadores del sistema. Podría considerarse que refleja la capacidad del ser humano en un único y amplio ámbito de actividad (por ejemplo, las tareas domésticas o la conducción en el mundo real).

- Generalización extrema: Describe los sistemas abiertos con la capacidad de abordar tareas completamente nuevas que sólo comparten puntos comunes abstractos con situaciones previamente encontradas, aplicables a cualquier tarea y dominio dentro de un amplio alcance. Esto podría caracterizarse como "adaptación a incógnitas desconocidas en una gama desconocida de tareas y dominios". Las formas biológicas de inteligencia (los humanos y posiblemente otras especies in- telligentes) son el único ejemplo de un sistema de este tipo en este momento.

A esta lista podríamos, en teoría, añadir una entrada más: La "universalidad", que extendería la "generalidad" más allá del ámbito de las tareas relevantes para los humanos, a cualquier tarea que pueda ser abordada de forma práctica dentro de nuestro universo (nótese que esto es diferente de "cualquier tarea en absoluto", tal y como se entiende en los supuestos del teorema No Free Lunch [98, 97]).



Es importante destacar que el espectro de la generalización descrito anteriormente parece reflejar la organización de las capacidades cognitivas de los seres humanos tal y como se establece en las teorías de la estructura de la inteligencia en la psicología cognitiva.

\section{Entornos de evaluación con tareas conocidas}\label{section:state-of-the-art:evaluation-enviroments-with-know-tasks}

Por lo general utilizan amplias baterías de tareas de prueba para evaluar sistemas que buscan una mayor flexibilidad. En estos entornos los desarrolladores del sistema evaluado conocen de antemano el tipo de tareas que enfrentarán, solo que los escenarios varian tantos que resolver dichas tareas requerirán diferentes formas de "razonamiento" y "creatividad", expresion directa de la generalización a partir de su entrenamiento.

\subsection{Entorno de aprendizaje arcade:
Una plataforma de evaluacion general para agentes}\label{subsection:state-of-the-art:evaluation-enviroments-with-know-tasks:arcade-enviroment}

\subsection{El Proyecto MalmO}\label{subsection:state-of-the-art:evaluation-enviroments-with-know-tasks:project-malmO}

\subsection{El Behavior Suite}\label{subsection:state-of-the-art:evaluation-enviroments-with-know-tasks:bsuite}

\subsection{GLUE}\label{subsection:state-of-the-art:evaluation-enviroments-with-know-tasks:the-glue}


La lógica subyacente de estos esfuerzos es medir algo más general que la habilidad en una tarea específica ampliando el conjunto de tareas objetivo. Sin embargo, cuando se trata de evaluar la flexibilidad, un defecto crítico de estos puntos de referencia multitarea es que el conjunto de tareas sigue siendo conocido de antemano por los desarrolladores de cualquier sistema de realización de pruebas, y se espera que los sistemas de realización de pruebas sean capaces de practicar específicamente para las tareas objetivo, aprovechar el conocimiento previo incorporado específico de la tarea heredado de los desarrolladores del sistema, aprovechar el conocimiento externo obtenido a través del preentrenamiento, etc.

\section{ARC}\label{section:state-of-the-art:arc}

Medir únicamente la destreza en una tarea determinada no es suficiente para medir la inteligencia, porque la destreza está fuertemente modulada por el conocimiento previo y la experiencia: los datos de entrenamiento ilimitados permiten a los experimentadores "comprar" niveles arbitrarios de destreza para un sistema, de forma que se enmascara el propio poder de generalización del sistema.
- citar openIA con sus sistemas para jugar Dota etc