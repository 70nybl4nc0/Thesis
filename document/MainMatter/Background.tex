\chapter{Estado del Arte}\label{chapter:state-of-the-art}

- ARC

[ OTROS ENTORNOS DE EVALUACION basados en Juegos. ]

[ Entornos de evaluacion para algoritmos de RL ]

- Necesidad de entornos de evaluación para algoritmos de RL.

La comunidad contemporánea de la IA todavía se inclina por evaluar la inteligencia comparando la destreza que muestran las IA y los humanos en tareas específicas, como los juegos de mesa y los videojuegos.


Entorno de Aprendizaje Arcade para el Refuerzo
agentes de aprendizaje [6]

El Proyecto MalmO [71]

El Behavior Suite [68]

El GLUE [95]

SuperGLUE [94] 


Medir únicamente la destreza en una tarea determinada no es suficiente para medir la inteligencia, porque la destreza está fuertemente modulada por el conocimiento previo y la experiencia: los datos de entrenamiento ilimitados permiten a los experimentadores "comprar" niveles arbitrarios de destreza para un sistema, de forma que se enmascara el propio poder de generalización del sistema.
- citar openIA con sus sistemas para jugar Dota etc