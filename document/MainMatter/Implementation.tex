\chapter{Implementación}\label{chapter:implementation}


\section{HERRAMIENTAS}

\subsection{The Unity Engine}

Unity es una plataforma de desarrollo 3D en tiempo real que consta de un motor de renderizado y de física, así como de una interfaz gráfica de usuario llamada Unity Editor. Unity ha recibido una amplia adopción en los sectores del juego, la arquitectura, la ingeniería y la construcción, el automóvil y el cine, y es utilizado por una gran comunidad de desarrolladores de juegos para realizar una gran variedad de simulaciones interactivas, que van desde pequeños juegos para móviles y navegadores hasta juegos de consola de alto presupuesto y experiencias de RA/VR.

\subsection{Unity ML-Agents Toolkit}

El Unity Machine Learning Agents Toolkit (ML-Agents) es un proyecto de código abierto que permite que los juegos y las simulaciones sirvan como entornos para el entrenamiento de agentes inteligentes. Proporcionamos implementaciones (basadas en PyTorch) de algoritmos de última generación para permitir a los desarrolladores de juegos y aficionados entrenar fácilmente agentes inteligentes para juegos 2D, 3D y VR/AR. Los investigadores también pueden utilizar la API de Python, que es muy fácil de usar, para entrenar a los agentes mediante el aprendizaje por refuerzo, el aprendizaje por imitación, la neuroevolución o cualquier otro método. Estos agentes entrenados pueden ser utilizados para múltiples propósitos, incluyendo el control del comportamiento de los NPCs (en una variedad de escenarios tales como multi-agentes y adversarios), pruebas automatizadas de construcciones de juegos y evaluación de diferentes decisiones de diseño de juegos antes de su lanzamiento. El ML-Agents Toolkit es mutuamente beneficioso tanto para los desarrolladores de juegos como para los investigadores de IA, ya que proporciona una plataforma central en la que los avances en IA pueden ser evaluados en los ricos entornos de Unity y luego se hacen accesibles a las comunidades más amplias de investigadores y desarrolladores de juegos.


- Visual Studio, que es y como funciona.

Las características del kit de herramientas incluyen un conjunto de entornos de ejemplo, algoritmos RL de última generación Soft Actor-Critic (SAC) (\cite{haarnoja2018soft}) y Proximal Policy Optimization (PPO) (\cite{schulman2017proximal})  entre otros muchos tipos de algoritmos.



- Creando las bases para los Agentes.

- Creando las bases para los Juegos.

- Implementación del ciclo de entrenamiento y de pruebas.

- Infiriendo listas de juegos.

- Añadiendo la evaluación a humanos.

- Ejemplo creando Agente 

- Ejemplo creando Juego