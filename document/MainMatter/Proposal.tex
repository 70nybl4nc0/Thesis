\chapter{Propuesta}\label{chapter:proposal}


Universal Test for Human oposing inteligen Agents (UTHOPIA)

ARC es muy bueno pero no se ajusta a los algoritmos RL.

Las tareas de proposito general capturadas en ARC carecen del concepto de tiempo. La figuracion implicita de direccion, cambios y movimiento de los humanos es dada gracias a la experiencia de estos y su capacidad de asociarla a eventos fisicos en los que está mas familiarizado.

Además ARC no es compatible para los algoritmos de RL, los cuales requieren de más interacción observacion - enviroment - accion.

Argumentamos que creemos necesario y mas justo potenciar las premisas teniendo en cuenta el efecto del tiempo. La persepcion explicita de los cambios y eventos. Y llevar las tareas a conceptos mas primitivos en terminos de aprendizaje sin descuidar la dificultad de generalizacion.

2.1 Que es UTHOPIA

 Descripcion y Objetivo

Es una prueba pensada tanto para humanos como para agentes inteligentes. Asumimos Core Knowlege basales.

El ambito de evaluacion se encuentra en la resolucion de tareas en forma de Juegos con las definiciones señaladas.

Premisas de conocimiento 

Trataremos de controlar los conocimientos previos que asumimos poseen los agentes.

**Premisas de objetividad**
- Cohesión de los objetos
- Persistencia de los objetos
- Influencia de los objetos a través del contacto
- Movimiento y dirección implícito en el cambio temporal

**Números y capacidades de conteo**

**Conocimientos básicas de geometría y topología**
Se debe reconocer la 
- Líneas y formas.
- Simetrías, rotaciones y traslaciones.
- Aumento o disminución de la forma, distorsiones elásticas.
- Contener / ser contenido / estar dentro o fuera de un perímetro.
- Alcanzar o permanecer en el borde de un perimetro.
- Trazado de líneas, conexión de puntos, proyecciones ortogonales.

- Memoria